\chapter*[INTRODUÇÃO]{Introdução}
\addcontentsline{toc}{chapter}{INTRODUÇÃO}

\section*{Motivação}
\addcontentsline{toc}{section}{Motivação}

Recentemente, houve a migração da força computacional dos computadores de mesa para dispositivos móveis. Dispositivos como câmeras e celulares podem adquirir e processar uma vasta quantidade de informação em curto intervalo de tempo. Apesar de conveniente, a troca de informações entre estes dispositivos se mantem um desafio devido ao seu pequeno tamanho e portabilidade. Conexões de alta velocidade são necessárias para permitir que a informação possa ser transferida de maneira eficaz e menos sofrível entre estes dispositivos portáteis e rede de computadores, servidores de armazenamento ou internet. 

Uma possível solução para a transferência de dados é uma conexão direta elétrica entre o dispositivo móvel e o servidor. Essa conexão elétrica por cabo e conectores nas duas extremidades oferece muitas desvantagens, pois os conectores podem ser caros devido ao pequeno tamanho do dispositivo, e mesmo assim esses cabos são propensos a falha e a quebra com o uso contínuo. 
Para o usuário esta solução apresenta o inconveniente de ter que utilizar cabos que limitam a mobilidade do dispositivo. E um dos meios de solucionar tal problema de transferência de dados se encontra na comunicação óptica sem fio.

A comunicação óptica sem fio é uma tecnologia que ainda se encontra em desenvolvimento, porém nos últimos cinquenta anos apresentou um avanço muito grande. Existem muitas vantagens neste tipo de comunicação em relação a comunicação via rádio e a sistemas de comunicação cabeados. A comunicação óptica sem fio oferece maior largura de banda e potencialmente maiores taxas de transmissão de dados se comparados com a comunicação via radio convencional.
O espectro de luz em que a comunicação óptica trabalha não é regulada, oferecendo maior flexibilidade quando comparada as frequências de rádio, as quais são regulamentadas. Não somente o espectro utilizado não é regulado, como a faixa de luz enxergada pelos humanos esta contida, permitindo que os dispositivos utilizados para transmissão, também possam ser utilizados para promover luz virtualmente constante ao ambiente enquanto transmite informações em alta frequência.\cite{Hranilovic}

Este método de comunicação que ainda está em fase de desenvolvimento e apresenta possibilidades de aplicação em diversas áreas da engenharia, facilitando a comunicação em localidades em que o uso de RF é restrita ou inviável. Como  por exemplo em bases aéreas,  hospitais, laboratórios e demais locais que deseja-se evitar a interferência das ondas de rádio em circuitos e equipamentos.
As aplicações do VLC variam de acordo com a distância de transmissão, podendo transmitir informações a distancias da ordem de nanômetros dentro de circuitos integrados e até mesmo realizar a comunicação entre estações base e satélites. 

Não obstante a tecnologia de comunicação óptica simplifica os circuitos de transmissão e recepção visto que não é necessário o dispendioso trabalho de projetar circuitos imunes a interferências das ondas eletromagnéticas.\cite{Hranilovic}


\section*{Objetivos}
\addcontentsline{toc}{section}{Objetivos}

Neste trabalho é apresentado uma alternativa aos sistemas de radiofrequência, utilizando a faixa de luz visível ao olho humano para realizar a transferência de dados. O nome dados a esse tipo de comunicação  é VLC.
Será confeccionado um protótipo funcional. E este dispositivo deve realizar a comunicação entre dois pontos, o transmissor e o receptor. O protótipo deve ser capaz de iluminar o ambiente e transmitir informações a um distancia de curto e médio alcance, ordem de metros, utilizando a tecnologia óptica.



\section*{Estrutura do texto}
\addcontentsline{toc}{section}{Estrutura do texto}

Este trabalho encontra-se organizado em três capítulos. Cada capítulo é voltado a uma parte específica do trabalho, partindo dos conhecimentos prévios do sistema a ser montado, passando pela escolha de material e finalmente chegando ao teste do protótipo. 
\begin{itemize}
	\item Parte I : Apresenta o histórico e as possíveis aplicações da tecnologia óptica a ser desenvolvida.
	
	\item Parte II: Tem como objetivo fornecer os dados necessários ao projeto de hardware físico tanto do transmissor como do receptor.
	
	\item Parte III: É dedicada ao teste do sistema construído apresentando os dados teóricos obtidos.
	
	\item Apêndices: Esta seção inclui os códigos utilizados nos testes de transmissão e o cronograma proposto para a segunda parte do Trabalho de Conclusão de Curso.

\end{itemize}
